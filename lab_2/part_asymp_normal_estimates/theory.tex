\subsection{Доверительные интервалы для параметров нормального распределения}
	\subsubsection{Доверительный интервал для математического ожидания $m$ нормального распределения}
	Дана выборка ($x_{1},x_{2}, ... ,x_{n}$) объёма n из нормальной генеральной совокупности. На её основе строим выборочное среднее $\bar{x}$ и выборочное среднее квадратическое отклонение $s$. Параметры $m$ и $\sigma$ нормального распределения неизвестны.

    Доверительный интервал для $m$ с доверительной вероятностью $\gamma = 1-\alpha$:
    \begin{equation}
        \begin{split}
             P\left(\bar{x} - \frac{sx}{\sqrt{n-1}} < m <  \bar{x} + \frac{sx}{\sqrt{n-1}}\right) = 2F_{T}(x) - 1 = 1 - \alpha,  \\
             P\left(\bar{x} - \frac{st_{1-\alpha/2}(n-1)}{\sqrt{n-1}} < m <  \bar{x} + \frac{st_{1-\alpha/2}(n-1)}{\sqrt{n-1}}\right)= 1 - \alpha
        \end{split}
        \label{eq:P_m}
    \end{equation}

    \subsubsection{Доверительный интервал для среднего квадратического отклонения $\sigma$ нормального распределения}
    Дана выборка ($x_{1},x_{2}, ... ,x_{n}$) объёма n из нормальной генеральной совокупности. На её основе строим выборочную дисперсию $s^{2}$. Параметры $m$ и $\sigma$ нормального распределения неизвестны.

    Задаёмся уровнем значимости $\alpha$.

    Доверительный интервал для $\sigma$ с доверительной вероятностью $\gamma = 1 - \alpha$:
    \begin{equation}
         P\left(\frac{s\sqrt{n}}{\sqrt{\chi^{2}_{1-\alpha/2}(n-1)}} < \sigma <  \frac{s\sqrt{n}}{\sqrt{\chi^{2}_{\alpha/2}(n-1)}}\right) = 1- \alpha,
         \label{eq:fin_interval}
    \end{equation}

    \subsection{Доверительные интервалы для математического ожидания $m$ и среднего квадратического отклонения $\sigma$ произвольного распределения при большом объёме выборки. Асимптотический подход}
    При большом объёме выборки для построения доверительных интервалов может быть использован асимптотический метод на основе центральной предельной теоремы.
    \subsubsection{Доверительный интервал для математического ожидания $m$ произвольной генеральной совокупности при большом объёме выборки}
    Предполагаем, что исследуемое генеральное распределение имеет конечные математическое ожидание $m$ и дисперсию $\sigma^{2}$.

    $u_{1-\alpha/2}$ — квантиль нормального распределения $N(0,1)$ порядка $1-\alpha/2$.

    Доверительный интервал для $m$ с доверительной вероятностью $\gamma = 1-\alpha$:
    \begin{equation}
        P\left(\bar{x} - \frac{su_{1-\alpha/2}}{\sqrt{n}} < m < \bar{x} - \frac{su_{1-\alpha/2}}{\sqrt{n}} \right) \approx \gamma,
        \label{eq:p_fin_u}
    \end{equation}

    \subsubsection{Доверительный интервал для среднего квадратического отклонения $\sigma$ произвольной генеральной совокупности при большом объёме выборки}
    Предполагаем, что исследуемая генеральная совокупность имеет конечные первые четыре момента.
    \newline
    $u_{1-\alpha/2}$ — корень этого уравнения — квантиль нормального распределения $N(0,1)$ порядка $1-\alpha/2$

    $E = \frac{\mu_{4}}{\sigma^{4}} - 3$ — эксцесс генерального распределения, e = $\frac{m_{4}}{s^{4}} - 3$ — выборочный эксцесс; $m_{4} = \frac{1}{n}\sum_{i =1}^{n}{(x_{i} - \bar{x})^{4}}$  - четвёртый выборочный центральный момент.

    \begin{equation}
        s(1 + U)^{-1/2} < \sigma < s(1-U)^{-1/2}
        \label{eq:sigma_int_1}
    \end{equation}
     или
    \begin{equation}
        s(1-0.5U) < \sigma < s(1 + 0.5U)
        \label{eq:sigma_int_2}
    \end{equation}
    где $U = u_{1-\alpha/2} \sqrt{(e + 2)/n}$

    Формулы (\ref{eq:sigma_int_1}) или \ref{eq:sigma_int_2} дают доверительный интервал для $\sigma$ с доверительной вероятностью $\gamma = 1-\alpha$.
    \newline
    \textit{Замечание.} Вычисления по формуле (\ref{eq:sigma_int_1}) дают более надёжный результат, так как в ней меньше грубых приближений.