\subsection{Проверка гипотезы о законе распределения генеральной совокупности. Метод хи-квадрат}

\subsubsection{Метод максимального правдоподобия}
\newline
$\hat{\mu} \approx 0.11, \hat{\sigma} \approx 0.99$
\newline
Критерий согласия $\chi^{2}$:
\newline
Количество промежутков $k = 8$
\newline
Уровень значимости $\alpha$= 0.05
\newline
Тогда квантиль $\chi^{2}_{1-\alpha}(k-1)$ = $\chi^{2}_{0.95}(7)$. Из таблицы [3, с. 358] $\chi^{2}_{0.95}(7) \approx 14.07$.
\begin{table}[H]
	\centering
	\begin{tabular}{| c | c | c | c | c | c | c |}
		\hline
		$i$ & $limits$         &   $n_i$ &    $p_i$ &   $np_i$ &   $n_i - np_i$ &   $\frac{(n_i-np_i)^2}{np_i}$ \\
		\hline
   1 & ['-inf', -1.1] &     9 & 0.1357 &  13.57 &        -4.57 &                        1.54 \\
   2 & [-1.1, -0.73]  &    12 & 0.096  &   9.6  &         2.4  &                        0.6  \\
   3 & [-0.73, -0.37] &     9 & 0.1253 &  12.53 &        -3.53 &                        0.99 \\
   4 & [-0.37, 0.0]   &    13 & 0.1431 &  14.31 &        -1.31 &                        0.12 \\
   5 & [0.0, 0.37]    &    15 & 0.1431 &  14.31 &         0.69 &                        0.03 \\
   6 & [0.37, 0.73]   &    14 & 0.1253 &  12.53 &         1.47 &                        0.17 \\
   7 & [0.73, 1.1]    &    16 & 0.096  &   9.6  &         6.4  &                        4.26 \\
   8 & [1.1, 'inf']   &    12 & 0.1357 &  13.57 &        -1.57 &                        0.18 \\
   $\sum$ & -            &   100 & 1      & 100    &        -0    &                        7.9  \\
		\hline
	\end{tabular}
	\caption{ Вычисление $\chi^{2}_{B}$ при проверке гипотезы $H_{0}$ о нормальном законе распределения $N(x,\hat{\mu}, \hat{\sigma})$}
	\label{tab:normal_chi_2}
\end{table}

\noindent Сравнивая $\chi^{2}_{B} = 7.9$ и $\chi^{2}_{0.95}(7) \approx 14.07$, видим, что $\chi^{2}_{B} < \chi^{2}_{0.95}(7)$.
Следовательно, на данном этапе гипотезу $H_{0}$ можно принять.
\\
\subsubsection{Исследование на чувствительность}
Рассмотрим выборки, распределённые по законам $L(x,0, \frac{1}{\sqrt{2}})$ и $U(x, -\sqrt{3}, \sqrt{3})$.
Проверим гипотезу о том, что они распределены по нормальному закону.

$\hat{\mu} \approx -0.03, \hat{\sigma} \approx 0.81$
\newline
Критерий согласия $\chi^{2}$:
\newline
Количество промежутков $k = 5$
\newline
Уровень значимости $\alpha$= 0.05
\newline
Тогда квантиль $\chi^{2}_{1-\alpha}(k-1)$ = $\chi^{2}_{0.95}(4)$. Из таблицы [3, с. 358] $\chi^{2}_{0.95}(4) \approx 9.49$.
\begin{table}[H]
	\centering
	\begin{tabular}{| c | c | c | c | c | c | c |}
		\hline
		$i$ & $limits$         &   $n_i$ &    $p_i$ &   $np_i$ &   $n_i - np_i$ &   $\frac{(n_i-np_i)^2}{np_i}$ \\
		\hline
   1 & ['-inf', -1.1] &     2 & 0.1357 &   2.71 &        -0.71 &                        0.19 \\
   2 & [-1.1, -0.37]  &     5 & 0.2213 &   4.43 &         0.57 &                        0.07 \\
   3 & [-0.37, 0.37]  &     8 & 0.2861 &   5.72 &         2.28 &                        0.91 \\
   4 & [0.37, 1.1]    &     4 & 0.2213 &   4.43 &        -0.43 &                        0.04 \\
   5 & [1.1, 'inf']   &     1 & 0.1357 &   2.71 &        -1.71 &                        1.08 \\
   $\sum$ & -            &    20 & 1      &  20    &        -0    &                        2.29 \\
		\hline
	\end{tabular}
	\caption{ Вычисление $\chi^{2}_{B}$ при проверке гипотезы $H_{0}$ о законе распределения $L(x,\hat{\mu}, \hat{\sigma})$, $n=20$}
	\label{tab:laplace_chi_2}
\end{table}

\noindent Сравнивая $\chi^{2}_{B} = 2.29$ и $\chi^{2}_{0.95}(4) \approx 9.49$, видим, что $\chi^{2}_{B} < \chi^{2}_{0.95}(4)$.
Следовательно, на данном этапе гипотезу $H_{0}$ можно принять.
\\



$\hat{\mu} \approx -0.34, \hat{\sigma} \approx 1.03$
\newline
Критерий согласия $\chi^{2}$:
\newline
Количество промежутков $k = 5$
\newline
Уровень значимости $\alpha$= 0.05
\newline
Тогда квантиль $\chi^{2}_{1-\alpha}(k-1)$ = $\chi^{2}_{0.95}(4)$. Из таблицы [3, с. 358] $\chi^{2}_{0.95}(4) \approx 9.49$.
\begin{table}[H]
	\centering
	\begin{tabular}{| c | c | c | c | c | c | c |}
		\hline
		$i$ & $limits$         &   $n_i$ &    $p_i$ &   $np_i$ &   $n_i - np_i$ &   $\frac{(n_i-np_i)^2}{np_i}$ \\
		\hline
   1 & ['-inf', -1.1] &     6 & 0.1357 &   2.71 &         3.29 &                        3.98 \\
   2 & [-1.1, -0.37]  &     7 & 0.2213 &   4.43 &         2.57 &                        1.5  \\
   3 & [-0.37, 0.37]  &     3 & 0.2861 &   5.72 &        -2.72 &                        1.3  \\
   4 & [0.37, 1.1]    &     1 & 0.2213 &   4.43 &        -3.43 &                        2.65 \\
   5 & [1.1, 'inf']   &     3 & 0.1357 &   2.71 &         0.29 &                        0.03 \\
   $\sum$ & -            &    20 & 1      &  20    &        -0    &                        9.46 \\
		\hline
	\end{tabular}
	\caption{ Вычисление $\chi^{2}_{B}$ при проверке гипотезы $H_{0}$ о законе распределения $U(x,\hat{\mu}, \hat{\sigma})$, $n=20$}
	\label{tab:uniform_chi_2}
\end{table}

\noindent Сравнивая $\chi^{2}_{B} = 9.46$ и $\chi^{2}_{0.95}(4) \approx 9.49$, видим, что $\chi^{2}_{B} < \chi^{2}_{0.95}(4)$.
Из-за близости значений следует провести дополнительные исследования.
\\
