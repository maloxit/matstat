\subsection{Ядерные оценки плотности распределения}
Рисунки, посвященные ядерным оценкам, иллюстрируют сближение ядерной оценки и функции плотности вероятности для всех $h$ с ростом размера
выборки. Для распределения Пуассона наиболее ярко видно, как сглаживает отклонения увеличение параметра сглаживания $h$.

В зависимости от особенностей распределений для их описания лучше подходят разные параметры $h$ в ядерной оценке: для равномерного распределения и распределения Пуассона лучше подойдет параметр $h = 2h_n$, для распределения Лапласа - $h = h_n/2$, а нормального и Коши - $h = h_n$.
Такие значения дают вид ядерной оценки наиболее близкий к плотности,
характерной данным распределениям

Также можно увидеть, что чем больше коэффициент при параметре сглаживания $\hat{h_n}$, тем меньше изменений знака производной у аппроксимирующей функции, вплоть до того, что при $h = 2h_n$ функция становится унимодальной на рассматриваемом промежутке. Также видно, что при $h = h_n / 2$
по полученным приближениям становится сложно сказать плотность вероятности какого распределения они должны повторять, так как они очень
похожи между собой.