\subsection{Оценки плотности вероятности}
	\subsubsection{Определение}
	\noindent Оценкой плотности вероятности $f(x)$ называется функция $\widehat{f}(x)$, построенная на основе выборки, приближённо равная $f(x)$
    \begin{equation}
        \widehat{f}(x)\approx f(x).
    \end{equation}
	\subsubsection{Ядерные оценки}
	\noindent Представим оценки в виде суммы с числом слагаемых, равным объёму выборки:\begin{equation}
        \widehat{f}_n(x)=\frac{1}{n h_n}\sum_{i=1}^n K\left(\frac{x-x_i}{h_n}\right).
    \end{equation}
    Здесь фукнция $K(u)$, называемая ядерной (ядром), непрерывна и является плотностью вероятности, $x_1,...,x_n$ $-$ элементы выборки, ${h_n}$ — любая последовательность положительных чисел, обладающая свойствами
    \begin{equation}
        h_n\xrightarrow[n\to\infty]{}0;\;\;\;\frac{h_n}{n^-1} \xrightarrow[n\to\infty]{}\infty.
    \end{equation}
    Такие оценки называются непрерывными ядерными.\\\\
    Гауссово (нормальное) ядро
    \begin{equation}
        K(u)=\frac{1}{\sqrt{2\pi}}e^{-\frac{u^2}{2}}.
    \end{equation}
    Правило Сильвермана
    \begin{equation}
        h_n=\left(\frac{4\hat{\sigma}^5}{3n}\right)^{1/5}\approx1.06\hat{\sigma}n^{-1/5},
    \end{equation}
    где $\hat{\sigma}$ - выборочное стандартное отклонение.