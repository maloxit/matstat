\subsection{Доля и теоретическая вероятность выбросов}
По данным, приведенным в таблице, можно сказать, что чем больше выборка, тем ближе доля выбросов будет к теоретической оценке. Снова доля выбросов для распределения Коши значительно выше, чем для остальных распределений. Равномерное распределение же в точности повторяет теоретическую оценку - выбросов мы не получали.
Боксплоты Тьюки действительно позволяют более наглядно и с меньшими усилиями оценивать важные характеристики распределений. Так, исходя из полученных рисунков, наглядно видно то, что мы довольно трудоёмко анализировали в предыдущих частях.