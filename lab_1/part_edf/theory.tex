\subsection{Эмпирическая функция распределения}
	\subsubsection{Статический ряд}
	\noindent Статистическим ряд- последовательность различных элементов выборки $z_1, z_2, ..., z_k$ положенных в возрастающем порядке с указанием частот $n_1, n_2, ..., n_k$, с которыми эти элементы содержатся в выборке. Обычно записывается в виде таблицы.
	\subsubsection{Эмпирическая функция распределения}
	\noindent Эмпирическая (выборочная) функция распределения (э.ф.р)- относительная частота события $X<x$, полученная по данной выборке:
	    \begin{equation}
            F_n^* = P^*(X<x)
        \end{equation}
	\subsubsection{ Нахождение эмпирической функции распределения}
	\noindent Для получения относительной частоты $P^*(X<x)$  просуммируем в статистическом ряде, построенном по данной выборке, все частоты $n_i$, для некоторых элементов $z_i$ статистического ряда меньше $x$. Тогда $P^*(X<x) = \frac{1}{n}\sum_{z_i<x}n_i$. Получаем
	\begin{equation}
        F^*(x)=\frac{1}{n}\sum_{z_i<x}n_i.
    \end{equation}
    $F^*(x)$-  функция распределения дискретной случайной величины $X^*$, заданной таблицей распределения
    \begin{table}[H]
    \centering
    \begin{tabular}{|c|c|c|c|c|}
        \hline
         $X^*$&$z_1$&$z_2$&...&$z_k$\\
         \hline
         $P$&$n_1/n$&$n_2/n$&...&$n_k/n$\\
         \hline
    \end{tabular}
    \caption{Таблица распределения}
    \label{tab:distr}
    \end{table}
    Эмпирическая функция распределения является оценкой, т. е. приближённым значением, генеральной функции распределения
    \begin{equation}
        F_n^*(x)\approx F_X(x).
    \end{equation}